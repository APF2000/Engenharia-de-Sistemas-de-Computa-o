\chapter{Introdução}\label{CAP:introducao}
%\thispagestyle{empty}

Este documento consiste de um modelo basico para a producao de documentos academicos, seguindo as normas ABNT. 

Nao e abordado o estudo do LaTex neste template. Sugerimos a leitura do texto em \citeonline{Oetiker:1995}. O uso do LaTex e aconselhavel devido a sua qualidade grafica, facil referenciacao, criacao de listas, indices, referencias bibliograficas e escrita matematica profissional. Porem, nao e obrigatorio o uso deste template, apenas as orientacoes de formatação segundo as normas ABNT devem ser obrigatoriamente seguidas.

Uma observação em particular é a de que, no pacote ABNTex, as referências diretas devem utilizar o comando ``citeonline''. Referências indiretas utilizam o comando ``cite''.

Exemplo de citacao direta: Uma otima fonte de estudo para compreender o LaTex e apresentada por \citeonline{Oetiker:1995}. 

Exemplo de citação indireta: Existem boas fontes de pesquisa para entendimento do LaTex \cite{Oetiker:1995}, estas incluem documentação online disponível na web.

\section{Objetivos}


 
\section{Diferenças UML e SysML}




\sectioan {Uso}


\section{História}

Com o intuito de elaborar uma linguagem unificada de propósito geral para engenharia de sistemas, diante das limitações da UML (Unified Modeling Language) a SysML foi criada pelo Object Management Group (OMG) em conjunto com o International Council on Systems Engineering (INCOSE). Em 2003, foram publicados os requisitos de uma linguagem de modelagem que servisse para Engenharia de Sistemas, criando-se, em seguida, um grupo de trabalho composto por representantes da indústria e produtores de ferramentas CASE chamado SysML Partners. Esse grupo ficou responsável por desenvolver a linguagem respeitando os requisitos estabelecidos. Em 2004 foi publicada a versão draft da SysML e em 2005 a versão SysML 1.0. A versão formal pública SysML OMG v 1.2 foi publicada pela  OMG em Junho de 2010. Desde então a linguagem vem se tornando cada vez disseminada e aceita pela comunidade se mostrando adequada para as demandas de Engenharia de Sistemas.
Assim surgiu a SysML, como uma extensão da UML V2 que expande a proposta de programação orientada a objetos possibilitando a representação de requisitos do sistema, componentes que não são de softwares, equações, fluxos contínuos e alocações.


\noindent \textbf{Capitulo \ref{CAP2}}: descricao...

\noindent \textbf{Capitulo \ref{CAP3}}: descricaoo...

\noindent \textbf{Capitulo \ref{CAP4}}: descricao...

\noindent \textbf{Capitulo \ref{CAP5}}: descricao...